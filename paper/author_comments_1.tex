\documentclass{article}
\usepackage[a4paper,width=150mm,top=25mm,bottom=25mm]{geometry}
\usepackage[natbib,style=apa,backend=biber]{biblatex}
\bibliography{author_comments_1.bib}
\usepackage{graphicx}
\usepackage[dvipsnames]{xcolor}


\begin{document}

\noindent{
  \textbf{\underline{KEY}}

  {
    \color{blue}
    Reviewer comments (blue)
  }

  Response (black)

  {
    \color{ForestGreen}
    New or changed text (green)
  }
}

\section*{Response to Martin Siegert's comments}


\noindent{\bf General comments}

\begin{quote}
\color{blue}
  I very much enjoyed looking at this paper.
  Using neural networks (and ai) to better depict the shape of the Antarctic bed is a great idea, and I applaud this effort.

  The authors have done a good job in describing their work, and its potential significance, and I think it should be published in the Cryosphere with some moderate changes first necessary.

  I like that this paper represents a new approach to studying the bed landscape in Antarctica and for that reason it should be a valuable asset for future work.

  There are a few ways it can be improved, however - and I note my comments in the attached pdf.
\end{quote}

We would like to thank the reviewer for their feedback, and for recognizing the significance of this work on applying Deep Learning to the Cryospheric domain.
Some interesting comments have been raised on the output and inner workings of the model, and we will respond to each individual comment in depth below.
It is nice to see that we are in agreement on several ideas, and that there is a clear path towards what is needed in terms of data collection to improve the next generation model.

\bigskip
\noindent{\bf Specific comments}

\begin{quote}
\color{blue}
  1. some discussion on the fact that Deepbed seems to be rougher than the base data.
\end{quote}

Correct, the DeepBedMap DEM does appear to be rougher than the base data (groundtruth) in Fig. 6 of the manuscript, and also in general, but this roughness is also something that can be adjusted by tweaking the training regime.
The DeepBedMap neural network model works by minimizing the elevation error between the groundtruth DEM and the predicted DeepBedMap DEM.
So the main product is bed elevation, with roughness being a secondary statistic derived from this generated bed elevation.
It is certainly possible to incorporate roughness (or any other statistical measure) into the loss function, to yield the desired surface, and this will be explored in future work.

{
  \color{ForestGreen}
  Added note on rougher bed and explanation at lines 297-300.
}

\begin{quote}
\color{blue}
  2. how roughness anisotropy is captured, as this is known to occur and should be critical to more accurate modelling.
\end{quote}

Bed roughness anisotropy is indeed an important consideration, and a good example is shown by \citet{HolschuhLinkingpostglaciallandscapes2020} who used swath radar to characterize elongated features (e.g. crag and tails) at the subglacial landscape of two sites in Thwaites Glacier.
We illustrate this over the same Thwaites Glacier region here in Fig 1, which shows DeepBedMap is able to capture aspects of the bed anisotropy from the groundtruth grid it was trained on (ice is flowing from top right to bottom left).

\iffalse
\begin{figure}[htbp]
  \includegraphics[width=0.95\textwidth]{figure-1_thwaites_glacier_anisotropy.png}
  \caption{
    Comparison of bed elevation grid products over Thwaites Glacier.
    Top - Groundtruth from gridded Operation IceBridge points.
    Middle - DeepBedMap.
    Bottom - BEDMAP2.
  }
  \label{fig:A}
\end{figure}
\fi

The DeepBedMap model derives bed anisotropy from 1) ice flow direction from the MEaSUREs ice velocity x and y components \citep{MouginotMEaSUREsPhaseMap2019}, 2) ice surface aspect derived from the REMA ice surface \citep{HowatReferenceElevationModel2019}, and 3) the BEDMAP2 bed elevation input \citep{FretwellBedmap2improvedice2013}.
There are therefore inherent assumptions that the topography of the current bed is associated with the current ice flow direction, surface aspect and existing BEDMAP2 anisotropy.
Provided that the direction of this surface velocity and aspect are the same as bed roughness anisotropy, as demonstrated in \citep{HolschuhLinkingpostglaciallandscapes2020}, the neural network will be able to recognize it and perform accordingly.
However, if the ice flow direction and surface aspect is not associated with bed anisotropy, then this assumption will be violated and the model will not perform well.

{
  \color{ForestGreen}
  Added new paragraph on how bed anisotropy is captured at lines 304-311.
}

\begin{quote}
\color{blue}
  3. how bed geology influences the roughness.
\end{quote}

While geology is linked to roughness, the training dataset does not adequately sample the distribution of different geology types over the Antarctica, nor is the the geology of Antarctica particularly well known beneath the ice.
Ideally, we would have a training dataset that is trained on different geological domains, and though the neural network does not currently take geology as an input, we see that this can be addressed in future work.
The main challenge lies in finding a suitable geological map (or geopotential proxy) with sufficient resolution and an adequate training dataset that covers the different lithologies.

To have geology as an input variable, we would ideally need to convert it from a lithological map (categorical/qualitative) to a hardness map with an appropriate erosion law and history incorporated (quantitative).
If the geology is given as a categorical variable (e.g. sedimentary, igneous or metamorphic), this may be harder to incorporate into neural networks that typically work with quantitative data.
Though it is possible to train Generative Adversarial Networks on qualitative data, it would require a more elaborate model architecture and loss function.

{
  \color{ForestGreen}
  Expanded section on how geology can be incorported in future studies at lines 328-334.
}

\begin{quote}
\color{blue}
  4. that there appear to be major gaps and to emphasize that radar is the only tool for solving this.
\end{quote}

Indeed, there is only so much we can extrapolate outside of the regions we have data for, no matter how advanced a technique we use.
Radio echo sounding is the best tool to not only provide the background coarse resolution dataset, but also the high resolution datasets needed for training.
Swath processing of existing datasets would be of great benefit.
Targeted acquisition of high resolution grids over a range of bed and flow types would also be beneficial.

{
  \color{ForestGreen}
  Emphasized importance of radar at lines 339-342.
}

\begin{quote}
\color{blue}
  5. importantly, that the approach could be better trained by working on formerly glaciated beds, such as the Laurentide ice sheet - or any land surface. Why not demonstrate the utility of the model in this way??
\end{quote}

Thank you for raising this idea.
We have actually considered this, though our thought was to use the swath bathymetry data around Antarctica instead.
The current model implementation does not support using solely 'elevation' as an input, as it also requires ice elevation, ice surface velocity and snow accumulation data.
To support using these paleo-beds as training data, one could do one of the following:

1. Have a paleo ice sheet model that provides these ice surface observation parameters.
However, continent scale ice sheet models quite often produce only kilometer scale outputs, and there are inherent uncertainties with past ice sheet reconstructions that may bias the resulting trained neural network model.

2. Modularize the neural network model to support different sets of training data.
It is theoretically possible to train one main branch with just the high resolution bed elevation data, and have the separate conditional inputs as optional branches into the model.
In fact, this main branch would simply be a Single Image Super Resolution problem, where we try to map a low resolution BEDMAP2 tile to a high resolution groundtruth image (be it from a contemporary bed, paleo bed, or offshore bathymetry).
The supporting conditional branches would then improve on the result of this naive super resolution method, and in particular, the ice velocity input would provide information on ice flow direction.
This modular neural network design would be more complicated to set up and train, but it will no doubt increase the available training data by at least an order of magnitude, and lead to better results.

{
  \color{ForestGreen}
  Added new paragraph on using formerly glaciated beds at lines 346-359.
}

\begin{quote}
\color{blue}
  That said, much of these issues can be addressed in future work.
  I still think this is a good piece of work and look forward to seeing the modified version.
\end{quote}

We hope this paper lays a foundation, and we too look forward to continuing this work and collaborating with others in the future.

\printbibliography

\end{document}
